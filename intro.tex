\section{Introduction}
Artificial Intelligence (AI) agents are revolutionizing software engineering by introducing autonomous, adaptive, and intelligent behaviors into the development and operation of complex systems. Early foundational works, such as Jennings' analysis of agent-based software engineering~\cite{jennings2000agent} and Wooldridge's exploration of agent-based approaches~\cite{wooldridge1997agent}, established the theoretical and practical significance of agents in modeling and implementing distributed systems. The evolution of agent-oriented methodologies~\cite{jennings1999agent} has further expanded the scope and impact of agents in software engineering.

Recent advances in large language models (LLMs) have led to the emergence of LLM-powered agents, which extend the capabilities of traditional agents by enabling sophisticated reasoning, tool use, and collaboration~\cite{liu2024llm}. These agents are increasingly being applied to automate and enhance various software engineering tasks, as highlighted in contemporary research~\cite{suri2023autonomous}. Together, these works demonstrate the transformative potential of AI agents in software engineering, while also identifying ongoing challenges and future research directions.

% References:
% jennings2000agent, wooldridge1997agent, jennings1999agent, liu2024llm, suri2023autonomous
